\documentclass[letterpaper, 12pt]{article}

\usepackage{amsthm,amssymb,amsmath}
\usepackage{enumerate}
\usepackage{fullpage}
\usepackage{flexisym}
\usepackage{graphicx}
\usepackage{mathtools}
\usepackage{tikz-cd,pgf}
\usepackage[normalem]{ulem}
\usepackage[bookmarks,colorlinks,breaklinks]{hyperref}  % PDF hyperlinks, with coloured links
\usepackage{color}
\definecolor{dullmagenta}{rgb}{0.4,0,0.4}   % #660066
\definecolor{darkblue}{rgb}{0,0,0.4}
\hypersetup{linkcolor=red,citecolor=blue,filecolor=dullmagenta,urlcolor=darkblue} % coloured links
%\hypersetup{linkcolor=black,citecolor=black,filecolor=black,urlcolor=black} % black links, for printed output
\usepackage{caption, subcaption}


\setlength{\textfloatsep}{10pt}
\setlength\parindent{0pt}
\setlength{\parskip}{0.5em}

\newcommand{\no}{\noindent}
\newcommand{\ud}{\,\mathrm{d}}
\newcommand{\sech}{\text{sech}}
\newcommand{\x}{{\mathbf x}}
\newtheorem{rmk}{Remark}
\newtheorem{thm}{Theorem}[subsection]
\newtheorem{cor}[thm]{Corollary}
\newtheorem{claim}{Claim}
\newtheorem{prop}[thm]{Proposition}
\newtheorem{lem}[thm]{Lemma}
\newtheorem{rem}[thm]{Remark}
\newtheorem{ques}[thm]{Question}
\newtheorem{definition}[thm]{Definition}
\newcommand{\ie}{{\it i.e. }}
\newcommand{\com}[1]{\color{blue}{ #1 }\color{black}}
\newcommand{\cl}[1]{{\mathrm{Cl}(#1)}}
\newcommand{\roi}[1]{\mathcal{O}_{#1}}
\newcommand{\gal}[3]{\mathrm{Gal}(#1 #2 #3)}
\newcommand{\iso}{\cong}
\newcommand{\rats}{\mathbb{Q}}
\newcommand{\reals}{\mathbb{R}}
\newcommand{\cmplx}{\mathbb{C}}
\newcommand{\Z}{\mathbb{Z}}
\newcommand{\Tone}{T_1}
\newcommand{\T}[1]{\Tone(#1)}
\newcommand{\Tvar}[1]{T_{#1}}
% produces a commutative square with punctuation at the end
\newcommand{\csq}[9]{\[
	\begin{tikzcd}[ampersand replacement=\&]
	{#1} \arrow{r}{#5} \arrow{d}{#6}
		\& {#2} \arrow{d}{#7} \\
	{#3} \arrow{r}{#8}
		\& {#4}
	\end{tikzcd}
#9\]}
\newcommand{\gl}[2]{\mathrm{GL}(#1, #2)}
\newcommand{\mat}[2]{\mathrm{M}(#1, #2)}
\newcommand{\inv}[1]{{#1}^{-1}}

\graphicspath{{./figures/}} %note trailing /

%\usepackage{showkeys}

\begin{document}  \section{Motivation, 8/24/16} Lie groups are differentiable
manifolds with a group structure. Tangent space of a Lie group has additional
structure, which we try to use to study the group action. Let $G, H$ be Lie
groups. Then we have \csq G H {\T G} {\T H} \varphi {} {} {\T \varphi}, where
$\Tone$ is the tangent space functor.

Recall that in the previous lecture, we consided the group $\gl n \reals$.
This is an open subspace of $\mat n \reals$ with $\dim \gl n \reals = \dim
\mat n \reals = n^2$. Now take a matrix $A$, and consider the curve passing
through the identity with tangent vector at $t = 0$ equal to $A$: $t
\rightarrow I + tA$. Thus we can identify tangent vectors with elements of
$\gl n \reals$.

If we look at the multiplication map $m$, we can look at \[\Tvar {(I, I)} (m)
: \T {\gl n \reals} \times \T {\gl n \reals} \rightarrow \T {\gl n \reals}.\]
Note that we can still speak of the functor $\Tone$ since a product of
manifolds is a manifold. Now consider the curve $t \rightarrow (I + tA, I +
tB)$, which has tangent vector $(A, B)$. Then $m(I + tA, I + tB) = I + t(A +
B) + t^2 AB.$ Hence the differential of $m$ is $A + B$, and $\T {(A, B)} = A +
B$, which is not very useful.

Take $A \in \gl n \reals$, and consider the map $B \rightarrow AB\inv A$. Let
$\mathrm{Int}(A)$ be this map, and let's calculate $\T {\mathrm{Int}(A)}.$ If
we look at \[ \mathrm{Int}(A)(I + tB) = I + tAB\inv A.\] The differential is
given by the same formula as the map, however they are different since the
domain/ranges aren't the same.

In some sense, the group operation is preserved, since we can tell whether the
group is abelian. Now look at $\mathrm{Int}(I + tA)(B) = (B + tAB)\inv {(I +
tA)}.$ Recall that the geometric series formula holds for matrices (for some
values of $t$), so we obtain the expansion $B + tAB - tBA + t^2 (\cdots).$ So
the differential is a map $(A, B) \rightarrow [A, B] = AB - BA.$ This gives us
an algebra structure, i.e. a Lie algebra.

\begin{definition} A Lie algebra $L$ over $k$ is a vector space with a Lie
bracket $L \times L \rightarrow L$ satisfying anticommutativity and the Jacobi
identity, \[ [A, [B, C]] + [B, [C, A]] + [C, [A, B]] = 0.\] \end{definition}

Anticommutativity implies that $[A, A] = 0$, unless the characteristic is
positive (which is the case for algebraic groups).

\begin{thm} Every finite-dimensional Lie algebra is isomorphic to a subalgebra
of a matrix algebra. \end{thm}

Some remarks about solvable and nilpotent Lie algebras. Comment about the Lie
algebra functor, which induces the diagram \csq G H {L(G)} {L(H)} \varphi {}
{} {L(\varphi)}, where $L(\varphi) := \T \varphi$.
 
Lie theory is not really able to see components. For instance, $\gl n \reals$
have two components, and by definition, only one of these has the identity. So
Lie algebra cannot ``see'' the components, and so it gives us something for
connected groups, e.g. $\T S^1 = \T \reals = \reals.$ However, it will be able
to tell us something about the group's universal cover.

\section{Overview, 8/26/16}
Recall we constructed the Lie group functor $G \rightarrow L(G)$ in the case of $G = \gl n \reals$ by looking at an inner automorphism.

Now assume $G$ is connected. If $G$ is nice enough, you can construct a universal covering space $\hat{G}$, and there is the covering map $p: \hat{G} \rightarrow G$. We will show that $p$ is a Lie group morphism, and that $L(p)$ is a Lie algebra isomorphism.

We will show that there is an equivalence of categories between Lie algebras and connected, simply-connected Lie groups. Take a connected Lie group $G$, $\dim G = 1$, so that $\dim L(G) = 1$ and $L(G) \iso \reals$ and $[ \cdot, \cdot ] = 0$ and $(\reals, +) = H$ has $L(H) = \reals$. Any such $G$ has $\reals$ as a covering space.Say we have a covering map $c: \reals \rightarrow C$. Then $\ker c$ is closed and a discrete subgroup so $G \iso \Z$ or $0$. So $G$ is either $\reals$ or $\reals/\Z$.

$G$ is a connected Lie group if and only if $L(G)$ is abelian, so the Lie bracket is identically zero. The rest is incomprehensible since I was sleepy.

\end{document}
