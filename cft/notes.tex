\documentclass[letterpaper, 12pt]{article}

\usepackage{amsthm,amssymb,amsmath}
\usepackage{enumerate}
\usepackage{fullpage}
\usepackage{flexisym}
\usepackage{graphicx}
\usepackage{mathtools}
\usepackage[bookmarks,colorlinks,breaklinks]{hyperref}  % PDF hyperlinks, with coloured links
\usepackage{color}
\definecolor{dullmagenta}{rgb}{0.4,0,0.4}   % #660066
\definecolor{darkblue}{rgb}{0,0,0.4}
\hypersetup{linkcolor=red,citecolor=blue,filecolor=dullmagenta,urlcolor=darkblue} % coloured links
%\hypersetup{linkcolor=black,citecolor=black,filecolor=black,urlcolor=black} % black links, for printed output
\usepackage{caption, subcaption}


\setlength{\textfloatsep}{10pt}
\setlength\parindent{0pt}
\setlength{\parskip}{0.5em}

\newcommand{\no}{\noindent}
\newcommand{\ud}{\,\mathrm{d}}
\newcommand{\sech}{\text{sech}}
\newcommand{\x}{{\mathbf x}}
\newtheorem{rmk}{Remark}
\newtheorem{thm}{Theorem}[subsection]
\newtheorem{cor}[thm]{Corollary}
\newtheorem{claim}{Claim}
\newtheorem{prop}[thm]{Proposition}
\newtheorem{lem}[thm]{Lemma}
\newtheorem{rem}[thm]{Remark}
\newtheorem{ques}[thm]{Question}
\newtheorem{definition}[thm]{Definition}
\newcommand{\ie}{{\it i.e. }}
\newcommand{\com}[1]{\color{blue}{ #1 }\color{black}}
\newcommand{\cl}[1]{{\mathrm{Cl}(#1)}}
\newcommand{\roi}[1]{\mathcal{O}_{#1}}
\newcommand{\gal}[3]{\mathrm{Gal}(#1 #2 #3)}
\newcommand{\iso}{\cong}
\newcommand{\rats}{\mathbb{Q}}
\renewcommand{\prime}{\mathfrak{p}}
\newcommand{\oprime}{\mathfrak{P}}
\newcommand{\oprimealt}{\oprime^o}
\newcommand{\frob}[1]{\mathrm{Fr}_{#1}}
\newcommand{\legndr}[2]{\left(\dfrac{#1}{#2}\right)}
\newcommand{\N}[1]{N(#1)}
\newcommand{\Z}[1]{\mathbb{Z}/#1 \mathbb{Z}}
\newcommand{\Zn}{\mathbb{Z}/n\mathbb{Z}}
\newcommand{\Znu}{\left(\mathbb{Z}/n\mathbb{Z}\right)^\times}
\newcommand{\Zp}{\mathbb{Z}/p\mathbb{Z}}
\newcommand{\Zpu}{\left(\mathbb{Z}/p\mathbb{Z}\right)^\times}

 \graphicspath{{./figures/}} %note trailing /

%\usepackage{showkeys}

\begin{document}

\section{Overview and Review, 8/23/16}

\subsection{Overview}
Class field theory provides a dictionary between abelian extensions of a given number field $F$ (\ie Galois extensions of $F$ with abelian Galois group) and intrinsic data about the number field, e.g. the class group of the ring of integers, $\cl{\roi F}$. For instance, if we let $H$ be the union of all abelian extensions of $F$ that are everywhere unramified, we have $\gal H/F \iso \cl{\roi F}$.

However, class field theory does not construct these abelian extensions, except for two classical constructions which were already known:
\begin{enumerate}
\item $F = \rats$, which has $H = \rats(\mu_\infty)$, where we have adjoined all the roots of unity. This is Kronecker-Weber theorem.
\item $F$ a quadratic imaginary extension of $\rats$.
\end{enumerate}
This is not so bad, however, since it turns out that class field theory can actually yield information about non-abelian extensions!

\subsection{Topics}
\begin{itemize}
\item Review local and global fields.
\item Group and Galois cohomology.
\item Local class field theory and local duality (important!).
\item Global class field theory and global duality.
\item Applications (Iwasawa theory).
\end{itemize}

\subsection{Review}
\subsubsection{First Example}
Let $L/K$ be a finite Galois extension. Let $\oprime$ be an unramified prime of $L$ lying over $\prime$, so that $\prime \roi L = \prod \oprime_i^{e_i}$ with all $e_i = 1$.
\begin{lem}
There is an element $\frob \oprime$ of $\gal L/K$ such that
\begin{enumerate}[1)]
\item $\frob \oprime (\oprime) = \oprime$, i.e. $\frob \oprime$ is in the decomposition group of $\oprime$.
\item \label{frauto} $\frob \oprime$ acts on $\roi L / \oprime$ as $x \rightarrow x^{N(\prime)}$.
\end{enumerate}
\end{lem}
\begin{rmk}
Note that $\roi K / \prime$ is a finite field of order $N(\prime)$, which has the Frobenius automorphism that does precisely what \ref{frauto} does. We should think of $\frob \oprime$ as a lift of that map to $\roi L/\oprime$. 
\end{rmk}

\begin{proof}
We will construct $\frob \oprime$ explicitly. Let $\alpha \in \roi L$ satisfying
\begin{enumerate}
\item $\alpha$ generates $(\roi L/\oprime)^{\times}$, and
\item for all $\oprimealt \neq \oprime$ above $\prime$, $\alpha \in \oprimealt$.
\end{enumerate}

Set $F(X) = \displaystyle \prod_{\sigma \in \gal L/K} (X - \sigma \alpha) \in \roi K[X].$ Then $F(\alpha) = 0$, so $F(\alpha^{\N \prime}) = F(\alpha)^{\N \prime} = 0$.

Then for some $\sigma, \alpha^{\N \prime} \equiv \sigma \alpha \pmod \oprime$. Then we claim that $\sigma \oprime = \oprime$. Otherwise, $\sigma^{-1} \oprime \neq \oprime$, so $\alpha \in \sigma^{-1} \oprime$, so $\sigma \alpha \in \oprime$. So $\alpha^{\N \prime} \equiv 0 \pmod \oprime$, a contradiction. 

Then for all $x \in \roi L / \oprime$, we can write $x = \alpha^i + b$, for some $i$ and $b \in \oprime$. Then \[\sigma(x) = \sigma(\alpha^i) + \sigma(b) = \alpha^{i \N \prime} + \sigma(b) \equiv x^{\N \prime} \pmod \oprime.\] Now define $\frob \oprime := \sigma$. Uniqueness is left to the reader. 
\end{proof}

\begin{rmk}
If $\oprimealt = \tau(\oprime)$ for some $\tau \in \gal L/K$, $\frob \oprimealt = \tau \frob \oprime \tau^{-1}$.
\end{rmk}

Recall that $L/K$ being Galois means $\gal L/K$ acts transitively on the primes $\oprime$ lying over $\prime$. So $\frob \oprime$ is well-defined up to conjugation. If $L/K$ is abelian, it is well-defined. 

\subsubsection{$\frob \oprime$ of Cyclotomic Field}
Let $L = \rats(\zeta_n)$ be the $n$-th cyclotomic field, $K = \rats$. Then $\gal L/K \iso \Znu$. Take $p$ unramified in $L/K$, i.e. $p$ not dividing $n$. Then $\frob p$ (we are in an abelian extension, so all $\frob \oprime$ are the same). By definition, $\frob p$ is the $\sigma$ such that $\sigma(\alpha) = \alpha^p \pmod \oprime$, for all $\oprime$ over $p$. 

Also characterized by $\tau(\zeta_n) = \zeta_n^p$ since \[\tau \sum a_i \zeta_n^i = (\sum a_i \zeta_n^i)^p.\]

\subsubsection{$\frob \oprime$ of Quadratic Field}
Here, we let $L = \rats(\sqrt{d})$ and $K$ as before. Then $\gal L/K \iso \Z 2$, and for $p$ unramified in $L$, $\frob p$ corresponds to $1$ if $p$ splits in $L$, or $-1$ if $p$ is inert in $L$. This means that \[\frob p = \legndr d p.\]

This connection leads us to an extremely nice proof of the quadratic reciprocity for odd primes.
\begin{thm}
Let $p \neq q$ be odd primes. Then $\legndr p q \legndr q p = (-1)^{\frac{p - 1}{2} \frac{q - 1}{2}}.$
\end{thm}
\begin{proof}
Let $L = \rats(\zeta_p)$, $K = \rats$. Then $\gal L/K$ is cyclic of order $p - 1$, and so has a unique order two quotient which corresponds to a quadratic field $F$, where $F = \rats\left(\sqrt{(-1)^{\frac{p - 1}{2}}p}\right)$.

Since $q$ is unramified, we can consider $\frob q$ which corresponds to $q \in \Zpu$. Now we simply compute this quantity in two ways.
\begin{enumerate}[(i)]
\item $\frob q \rvert_F = 1 \Leftrightarrow q^{\frac{p - 1}{2}} \equiv 1 \pmod p \Leftrightarrow \legndr q p = 1$.
\item $\frob q \rvert_F$ is also simply $\frob q$ for the quadratic extension $F$, hence equal to $\legndr{(-1)^{\frac{p - 1}{2}} p}{q}$ by the previous example.
\end{enumerate}
\end{proof}
\subsection{First Case of Fermat's Last Theorem}
\begin{thm}
If $p$ does not divide $|\cl {\rats(\zeta_p)}|$, then $x^p + y^p = z^p$ has no integer solutions with $p$ not dividing $xyz$.
\end{thm}

The idea is that we can factor $\prod_i (x + \zeta_p^i y) = z^p$ in $\mathbb{Z}[\zeta_p]$. It turns out that regularity gives us that the LHS factors are $p$-th powers, the divisibility condition giving us that the factors are coprime. 
\end{document}