\documentclass[letterpaper, 12pt]{article}

\usepackage{amsthm,amssymb,amsmath}
\usepackage{enumerate}
\usepackage{fullpage}
\usepackage{flexisym}
\usepackage{graphicx}
\usepackage{mathtools}
\usepackage[normalem]{ulem}
\usepackage[bookmarks,colorlinks,breaklinks]{hyperref}  % PDF hyperlinks, with coloured links
\usepackage{color}
\definecolor{dullmagenta}{rgb}{0.4,0,0.4}   % #660066
\definecolor{darkblue}{rgb}{0,0,0.4}
\hypersetup{linkcolor=red,citecolor=blue,filecolor=dullmagenta,urlcolor=darkblue} % coloured links
%\hypersetup{linkcolor=black,citecolor=black,filecolor=black,urlcolor=black} % black links, for printed output
\usepackage{caption, subcaption}


\setlength{\textfloatsep}{10pt}
\setlength\parindent{0pt}
\setlength{\parskip}{0.5em}

\newcommand{\no}{\noindent}
\newcommand{\ud}{\,\mathrm{d}}
\newcommand{\sech}{\text{sech}}
\newcommand{\x}{{\mathbf x}}
\newtheorem{rmk}{Remark}
\newtheorem{thm}{Theorem}[subsection]
\newtheorem{cor}[thm]{Corollary}
\newtheorem{claim}{Claim}
\newtheorem{prop}[thm]{Proposition}
\newtheorem{lem}[thm]{Lemma}
\newtheorem{rem}[thm]{Remark}
\newtheorem{ques}[thm]{Question}
\newtheorem{definition}[thm]{Definition}
\newcommand{\ie}{{\it i.e. }}
\newcommand{\com}[1]{\color{blue}{ #1 }\color{black}}
\newcommand{\cl}[1]{{\mathrm{Cl}(#1)}}
\newcommand{\roi}[1]{\mathcal{O}_{#1}}
\newcommand{\gal}[3]{\mathrm{Gal}(#1 #2 #3)}
\newcommand{\iso}{\cong}
\newcommand{\rats}{\mathbb{Q}}
\newcommand{\reals}{\mathbb{R}}
\newcommand{\cmplx}{\mathbb{C}}
\renewcommand{\prime}{\mathfrak{p}}
\newcommand{\oprime}{\mathfrak{P}}
\newcommand{\oprimealt}{\oprime^o}
\newcommand{\frob}[1]{\mathrm{Fr}_{#1}}
\newcommand{\legndr}[2]{\left(\dfrac{#1}{#2}\right)}
\newcommand{\N}[1]{N(#1)}
\newcommand{\Zmod}[1]{\mathbb{Z}/#1 \mathbb{Z}}
\newcommand{\Z}{\mathbb{Z}}
\newcommand{\Zn}{\mathbb{Z}/n\mathbb{Z}}
\newcommand{\Znu}{\left(\mathbb{Z}/n\mathbb{Z}\right)^\times}
\newcommand{\Zp}{\mathbb{Z}/p\mathbb{Z}}
\newcommand{\Zpu}{\left(\mathbb{Z}/p\mathbb{Z}\right)^\times}
\newcommand{\nnreals}{\mathbb{R}^{\geq 0}}
\newcommand{\Q}{\mathbb{Q}}
\newcommand{\Qp}{\Q_p}

 \graphicspath{{./figures/}} %note trailing /

%\usepackage{showkeys}

\begin{document}

\section{Overview and Review, 8/23/16}

\subsection{Overview} Class field theory provides a dictionary between abelian
extensions of a given number field $F$ (\ie Galois extensions of $F$ with
abelian Galois group) and intrinsic data about the number field, e.g. the
class group of the ring of integers, $\cl{\roi F}$. For instance, if we let
$H$ be the union of all abelian extensions of $F$ that are everywhere
unramified, we have $\gal H/F \iso \cl{\roi F}$.

However, class field theory does not construct these abelian extensions,
except for two classical constructions which were already known:
\begin{enumerate} 
\item $F = \rats$, which has $H = \rats(\mu_\infty)$, where
we have adjoined all the roots of unity. This is Kronecker-Weber theorem.
\item $F$ a quadratic imaginary extension of $\rats$.
\end{enumerate}
This is not so bad, however, since it turns out that class field theory can
actually yield information about non-abelian extensions!

\subsection{Topics}
\begin{itemize}
\item Review local and global fields.
\item Group and Galois cohomology.
\item Local class field theory and local duality (important!).
\item Global class field theory and global duality.
\item Applications (Iwasawa theory).
\end{itemize}

\subsection{Review}
\subsubsection{First Example}
Let $L/K$ be a finite Galois extension. Let $\oprime$ be an unramified prime of $L$
lying over $\prime$, so that $\prime \roi L = \prod \oprime_i^{e_i}$ with all $e_i = 1$.
\begin{lem}
There is an element $\frob \oprime$ of $\gal L/K$ such that
\begin{enumerate}[1)]

\item $\frob \oprime (\oprime) = \oprime$, i.e. $\frob \oprime$ is in the
decomposition group of $\oprime$.
\item \label{frauto} $\frob \oprime$ acts on $\roi L / \oprime$ as
 $x \rightarrow x^{N(\prime)}$.

\end{enumerate}
\end{lem}
\begin{rmk}

Note that $\roi K / \prime$ is a finite field of order $N(\prime)$, which has
the Frobenius automorphism that does precisely what \ref{frauto} does. We
should think of $\frob \oprime$ as a lift of that map to $\roi L/\oprime$.

\end{rmk}

\begin{proof}
We will construct $\frob \oprime$ explicitly. Let $\alpha \in \roi L$ satisfying
\begin{enumerate}
\item $\alpha$ generates $(\roi L/\oprime)^{\times}$, and
\item for all $\oprimealt \neq \oprime$ above $\prime$, $\alpha \in \oprimealt$.
\end{enumerate}

Set $F(X) = \displaystyle \prod_{\sigma \in \gal L/K} (X - \sigma \alpha) \in
\roi K[X].$ Then $F(\alpha) = 0$, so $F(\alpha^{\N \prime}) = F(\alpha)^{\N
\prime} = 0$.

Then for some $\sigma, \alpha^{\N \prime} \equiv \sigma \alpha \pmod \oprime$.
Then we claim that $\sigma \oprime = \oprime$. Otherwise, $\sigma^{-1} \oprime
\neq \oprime$, so $\alpha \in \sigma^{-1} \oprime$, so $\sigma \alpha \in
\oprime$. So $\alpha^{\N \prime} \equiv 0 \pmod \oprime$, a contradiction.

Then for all $x \in \roi L / \oprime$, we can write $x = \alpha^i + b$, for
some $i$ and $b \in \oprime$. Then \[\sigma(x) = \sigma(\alpha^i) + \sigma(b)
= \alpha^{i \N \prime} + \sigma(b) \equiv x^{\N \prime} \pmod \oprime.\] Now
define $\frob \oprime := \sigma$. Uniqueness is left to the reader.
\end{proof}

\begin{rmk}

If $\oprimealt = \tau(\oprime)$ for some $\tau \in \gal L/K$, $\frob
\oprimealt = \tau \frob \oprime \tau^{-1}$.

\end{rmk}

Recall that $L/K$ being Galois means $\gal L/K$ acts transitively on the
primes $\oprime$ lying over $\prime$. So $\frob \oprime$ is well-defined up to
conjugation. If $L/K$ is abelian, it is well-defined.

\subsubsection{$\frob \oprime$ of Cyclotomic Field}

Let $L = \rats(\zeta_n)$ be the $n$-th cyclotomic field, $K = \rats$. Then
$\gal L/K \iso \Znu$. Take $p$ unramified in $L/K$, i.e. $p$ not dividing $n$.
Then $\frob p$ (we are in an abelian extension, so all $\frob \oprime$ are the
same). By definition, $\frob p$ is the $\sigma$ such that $\sigma(\alpha) =
\alpha^p \pmod \oprime$, for all $\oprime$ over $p$.

Also characterized by $\tau(\zeta_n) = \zeta_n^p$ since \[\tau \sum a_i
\zeta_n^i = (\sum a_i \zeta_n^i)^p.\]

\subsubsection{$\frob \oprime$ of Quadratic Field}

Here, we let $L = \rats(\sqrt{d})$ and $K$ as before. Then $\gal L/K \iso \Zmod
2$, and for $p$ unramified in $L$, $\frob p$ corresponds to $1$ if $p$ splits
in $L$, or $-1$ if $p$ is inert in $L$. This means that \[\frob p = \legndr d
p.\]

This connection leads us to an extremely nice proof of the quadratic reciprocity
for odd primes.

\begin{thm}
Let $p \neq q$ be odd primes. Then $\legndr p q \legndr q p = (-1)^{\frac{p -
1}{2} \frac{q - 1}{2}}.$
\end{thm}

\begin{proof}

Let $L = \rats(\zeta_p)$, $K = \rats$. Then $\gal L/K$ is cyclic of order $p -
1$, and so has a unique order two quotient which corresponds to a quadratic
field $F$, where $F = \rats\left(\sqrt{(-1)^{\frac{p - 1}{2}}p}\right)$.

Since $q$ is unramified, we can consider $\frob q$ which corresponds to
$q \in \Zpu$. Now we simply compute this quantity in two ways.
\begin{enumerate}[(i)]

\item $\frob q \rvert_F = 1 \Leftrightarrow q^{\frac{p - 1}{2}} \equiv 1
\pmod p \Leftrightarrow \legndr q p = 1$.

\item $\frob q \rvert_F$ is also simply $\frob q$ for the quadratic
extension $F$, hence equal to $\legndr{(-1)^{\frac{p - 1}{2}} p}{q}$
by the previous example.

\end{enumerate}
\end{proof}
\subsection{First Case of Fermat's Last Theorem}
\begin{thm}

If $p$ does not divide $|\cl {\rats(\zeta_p)}|$, then $x^p + y^p = z^p$ has no
integer solutions with $p$ not dividing $xyz$.

\end{thm}

The idea is that we can factor $\prod_i (x + \zeta_p^i y) = z^p$ in
$\mathbb{Z}[\zeta_p]$. It turns out that regularity gives us that the LHS
factors are $p$-th powers, the divisibility condition giving us that the
factors are coprime.

\begin{proof}

We take $p > 5$, since we can easily prove the cases $p = 3,
5$ by looking at the equation modulo $9, 25$ respectively. Without loss of
generality, assume $x, y, z$ are coprime and $p$ does not divide $x - y$. If
$x \equiv y \pmod p$ and $x \equiv -z \pmod p$, then $-2z^p \equiv z^p \pmod
p$, a contradiction. So we must have one or the other.

First, prove the coprimeness of the factors. If a prime $q$ of
$\mathbb{Z}[\zeta_p]$ divides two factors $x + \zeta_p^k y$ for $k = i, j, i
\neq j$. Then $q | (\zeta_p^i - \zeta_p^j) y$. Since $p$ does not divide
$y$, and $q | (\zeta_p^j - \zeta_p^i) x$, so $q | (\zeta_p^i - \zeta_p^j)$,
the unique prime ideal over $p$. (Recall that $p \mathbb{Z}[\zeta_p] = (1 -
\zeta_p)^{p - 1}$). So $q = (1 - \zeta_p) = p$, so $p | x + y$, i.e. $x + y
\in \prime \cap \mathbb{Z} = p\mathbb{Z}$, so $x^p + y^p \equiv x + y \equiv
0 \pmod p$.

Hence, $(x + \zeta_p^i y) = J_i^p$ for some ideal $J_i$. But since $p$ does
not divide the class group, no element can have order $p$, hence $J_i$ is
principal. Then $x + \zeta_p^i y = u \alpha_i^p$ for some unit $u$.
Dirichlet's unit theorem doesn't give us precise control over what the units
look like, so we need to muck around with the following lemma.

\begin{lem}

Let $u$ be a unit. Then $u = \zeta_p^r \epsilon$ for a unit $\epsilon$ of the
maximal real subfield of the $p$-th cyclotomic field.

\end{lem}
\begin{proof}

Consider $u / \bar{u}$, an algebraic integer with norm 1 in all complex
embeddings. This means $u/\bar{u}$ is a root of unity, so \[u/\bar{u} = \pm
\zeta_p^s,\] for some $s$. If the sign is plus, let $r$ be such that $2r
\equiv s \pmod p$. It turns out that a minus sign gives us a contradiction.

\end{proof}

So $x + \zeta_p^i y = \zeta_p^r \epsilon \alpha^p$. Conjugation gives us \[x +
\zeta_p^i y \equiv \zeta^r \epsilon \alpha \] and \[ x + \zeta_p^{-i} y \equiv
\zeta^{-r} \epsilon \alpha \] modulo $p \mathbb{Z}[\zeta_p]$. Hence \[
\zeta_p^{-r} ( x + \zeta_p^i y) \equiv \zeta^r ( x + \zeta_p^{-i} y)
\pmod{p\mathbb{Z}[\zeta_p]}. \] We can conclude the proof from the following
lemma.

\begin{lem}

If $\alpha = a_0 + \cdots + a_{p - 1} \zeta_p^{p - 1}$, $a_i \in \mathbb{Z}$
not all zero, and if $\alpha \in m \mathbb{Z}[\zeta_p], m \in \mathbb{Z}$,
then $m | a_i$ for all $i$.

\end{lem}
\end{proof}

\section{Review of Local Fields, 8/25/16}
\subsection{Locla Fields Facts}
Let $K$ be a field, then absolute value on $K$ is a map to $\nnreals$ that is multiplicative, satisfies the triangle inequality and is positive semi-definite. Includes for instance the $p$-adic norm, archimedean absolute values.

\begin{definition}
Absolute value is \uline{non-archimedean} if $|x + y| \leq \max(|x|, |y|)$.
\end{definition}

In non-archimedean fields, we talk about valuations, i.e. maps $v: K \rightarrow \mathbb{R} \cup \{ \infty \}$ which is additive, i.e. $v(x + y) \geq \min( v(x), v(y))$ and $v(0) = \infty$. For instance, $v(p^n \frac{a}{b}) = n$.

\begin{definition}
Given a non-archimedean valuation, let
\begin{align*}
  \roi v & = \{ x \in k | |x| \leq 1 \} \\
  \mathfrak{m}_v & = \{ x \in k | |x| < 1 \} \\
  {\roi v}^\times & = \{ x \in k | |x| = 1 \}
\end{align*}
\end{definition}

\begin{lem}
$| \cdot |_v$ is discrete if and only if $\mathfrak{m}_v$ is principal.
\end{lem}

For example, $\Qp, | \cdot |_p$ is discrete, and $|\Qp \setminus 0| = p^\Z$.
However, $\overline \Qp$ with $| \cdot |_p$ is non-discrete, since
$|p^{1/n}| = p^{-1/n} \rightarrow 1$.

\begin{rmk}
Any $| \cdot |$ on $k$ topologizes $k$.
\end{rmk}

\begin{lem}
If $| \cdot |, | \cdot |'$ are absolute values on $k$, then TFAE:
\begin{enumerate}[(1)]
  \item they define the same topology
  \item $\roi {| \cdot |} = \roi {| \cdot |'}$
  \item $|x| = (|x|')^r$, some $r \in R$.
\end{enumerate}
\end{lem}

\begin{thm}[Ostrowski]
The only absolute values on $\Q$ are the usual one and the $p$-adic ones.
\end{thm}

\begin{thm}[Weak Approximation]
  Let $| \cdot |_1, ..., | \cdot |_n$ be pairwise inequivalent absolute values on $k$. Then \[ k \xhookrightarrow{} \prod_{i = 1}^n k_{| \cdot |_i} \] is dense, where the subscripts indicate completions with respect to those absolute values.
\end{thm}

\begin{proof}
First, we claim that there exists $a \in k$ with $|a|_1 > 1$ and $|a|_i < 1$ for the other $i$. Work by induction. This is clearly true for $n = 1$. For $n = 2$, if $| \cdot |_1 \not \iso | \cdot |_2$ then there are $b, c$ such that $|b|_1 < 1, |b|_2 \geq 1$, $|c|_1 \geq 1, |c|_2 < 1$. Then let $a = b/c$. By indcution, there's $b \in k$ such that $|b|_1 > 1, |b|_i < 1$ and (by $n = 2$ case), $|c| > 1, |c|_n < 1$. If $|b|_n < 1$, we're done. If $|b|_n = 1$, set $a = cb^m$ for $m$ large enough. Otherwise, set $a = cb^m/(1 + b^m)$. Done.

Next, we show that for any $\epsilon > 0$, there is an $\alpha \in k$ such that $|a - 1|_1 < \epsilon$ and $|a|_i < \epsilon$. Take $a$ as above, i.e. $\alpha^m/(1 + \alpha^m)$ for $m >> 0$. 

Let $x_1, ..., x_n \in k$. For each $i$, choose $|\alpha_i - 1|_i$ very small, $|\alpha_i|_{j \neq i}$ very small, and set $x = \sum x_i \alpha_i$.
\end{proof}

\subsection{Hensel's Lemma}
Let $\mathcal{O}$ be a complete discrete valuation ring (e.g. any local ring). We want to discuss Hensel's lemma.

\begin{thm}[Version 1]
  Let $f(x) \in \mathcal{O}[X]$. Let $a_0 \in \mathcal{O}$ be such that $|f(a_0)| < |f'(a_0)|^2$. Then there is a root $a \in \mathcal{O}$ of $f$ such that $|a - a_0| < |f(a_0)/f'(a_0)^2|$.
\end{thm}
\begin{proof}
Literally Newton's method.
\end{proof}

For example, if $\mathcal{O} = \Z_5$ and $f(X) = X^3 + X + 3$, then $f(1) \equiv 0 \pmod 5$, $f'(1) \equiv 4 \pmod 5$. Then we can let $a_1 = a_0 - f(a_0)/f'(a_0) \equiv 6 \pmod 25$.

\textbf{Important Example}. If $\mathcal{O} = \Zp$, and $f(X) = X^p - X$, then for any $a \in \mathbb{F}_p$, $f(a) \equiv 0 \pmod p$, and $f'(a) \equiv -1 \pmod p$. Then there exists an $a* \in \Zp$ such that $a* \equiv a \pmod p, f(a) = 0$. Hence we can conclude that the $p$-th roots of unity are contained in $\Qp$ and reduction mod $p$ is a group isomorphism from $\mu_{p - 1}(\Zp) \iso \mathbb{F}_p^\times$. The inverse of this map is called the Teichmuller lift.

In general, if $\mathcal{O}$ is a discrete valuation ring, with finite residue field $k$, then $\mu_{|k| - 1}(\mathcal{O}) \iso k^\times$.

\begin{thm}[Version 2]
  Let $\mathcal{O}$ be a DVR with residue field $k = \mathcal{O}/\mathfrak{m}$. Let $f(X) \in \mathcal{O}[X]$. If $\bar{f}(X) \in k[X]$ factors as $g_0(X) h_0(X)$ with $g_0, h_0$ monic and coprime, then $f(X) = g(X) h(X)$ for unique monic $g, h \in \mathcal{O}[X]$ such that $g \equiv g_0, h \equiv h_0 \pmod{\mathfrak{m}}$.
\end{thm}

\subsection{Extension of Valuations}
\begin{prop}
Let $k$ be complete with respect to a discrete $| \cdot |_v$. Let $L/K$ be a finite separable extension. Then $| \cdot |_v$ extends uniquely to $L$, $L$ is complete with respect to this extension, and for every $\alpha \in L$, $|\alpha|_v := |N_{L/K}(\alpha)|^{1/[L:K]}.$
\end{prop}

\begin{proof}
If both were an extension, then since all norms on a finite dimensional vector space over a complete field are equivalent, they define the same topology and so the exponent must be equal.

  To prove existence, note that there is a unique maximal ideal $\mathfrak{m}_L$ of $\mathcal{O}_L$ above that of $\roi k$. You can define a valuation upstairs by defining one on the maximal ideal, then normalize.
\end{proof}
\begin{rmk}
The formula is forced by existence of uniqueness. If $L/K$ is a Galois extension, $|\sigma( \cdot )|$ is another absolute value, so $| \sigma (\cdot)| = | \cdot |$ by uniqueness, and take the product over all elements of the Galois group.
\end{rmk}

Suppose $k$ is complete with respect to the discrete, non-archimedean valuation. Let $L/K$ be a finite separable extension. Let $v_k$ be the normalized valuation on $k$, i.e. $v: k \twoheadrightarrow \mathbb{Z}$. Let $\pi_k$ be the generator of the maximal ideal of $\roi k$ (a uniformizing element). Let $w : L \rightarrow \reals$ be the unique extension of $v_k$ to $L$.

\begin{definition}
  Let $e_{L/K} = [ w(L^\times) : v_k(k^\times)]$, i.e. $\pi_k \roi L = \pi_L^{e_{L/K}}$, be the \uline{ramification} degree. Let $f_{L/K} = [k_L : k_k]$ be the degree of the residue field extension be called the \uline{inertial} degree.
\end{definition}

\textbf{Example}. If $L = \Qp(\sqrt{p})$, $e_{L/K} = 2, f_{L/K} = 1$.

\textbf{Example}. Let $L = \Qp(\zeta)$ where $\zeta^{p^2 - 1} = 1$, and $\zeta^{p - 1} \neq 1$. Then $e_{L/K} = 1, f_{L/K} = 2$. Minimal polynomial is $(X - \zeta)(X - \zeta^p)$.

\begin{definition}[Purist's]
A local field is a field with nontrivial absolute value inducing a locally compact topology on $k$.
\end{definition}

\begin{definition}
Namely, it is either $\reals, \cmplx$ in the archimedean case, or in the non-archimedean case, a finite extension of $\Qp$ or $\mathbb{F}_p((T))$.
\end{definition}

Note that $k$ is complete with respect to non-archimedean absolute value if and only if the residue field is finite and $\roi K$ is compact. Looking ahead, we aim to understand $\gal \bar{k} k$ where $k/\Qp$ is a finite extension.
\end{document}
