\documentclass[letterpaper, 12pt]{article}

\usepackage{amsthm,amssymb,amsmath}
\usepackage{enumerate}
\usepackage{fullpage}
\usepackage{flexisym}
\usepackage{graphicx}
\usepackage{mathtools}
\usepackage[normalem]{ulem}
\usepackage[bookmarks,colorlinks,breaklinks]{hyperref}  % PDF hyperlinks, with coloured links
\usepackage{color}
\definecolor{dullmagenta}{rgb}{0.4,0,0.4}   % #660066
\definecolor{darkblue}{rgb}{0,0,0.4}
\hypersetup{linkcolor=red,citecolor=blue,filecolor=dullmagenta,urlcolor=darkblue} % coloured links
%\hypersetup{linkcolor=black,citecolor=black,filecolor=black,urlcolor=black} % black links, for printed output
\usepackage{caption, subcaption}


\setlength{\textfloatsep}{10pt}
\setlength\parindent{0pt}
\setlength{\parskip}{0.5em}

\newcommand{\no}{\noindent}
\newcommand{\ud}{\,\mathrm{d}}
\newcommand{\sech}{\text{sech}}
\newcommand{\x}{{\mathbf x}}
\newtheorem{rmk}{Remark}
\newtheorem{thm}{Theorem}[subsection]
\newtheorem{cor}[thm]{Corollary}
\newtheorem{claim}{Claim}
\newtheorem{prop}[thm]{Proposition}
\newtheorem{lem}[thm]{Lemma}
\newtheorem{rem}[thm]{Remark}
\newtheorem{ques}[thm]{Question}
\newtheorem{definition}[thm]{Definition}
\newcommand{\ie}{{\it i.e. }}
\newcommand{\com}[1]{\color{blue}{ #1 }\color{black}}
\newcommand{\cl}[1]{{\mathrm{Cl}(#1)}}
\newcommand{\roi}[1]{\mathcal{O}_{#1}}
\newcommand{\gal}[3]{\mathrm{Gal}(#1 #2 #3)}
\newcommand{\iso}{\cong}
\newcommand{\rats}{\mathbb{Q}}
\newcommand{\reals}{\mathbb{R}}
\newcommand{\cmplx}{\mathbb{C}}
\newcommand{\polyring}[3]{#1[x_{#2}, ..., x_{#3}]}
\newcommand{\graded}[2]{\bigoplus_{d = 0}^\infty \polyring{#1}{0}{#2}_d}
\newcommand{\ffield}[1]{\mathbb{F}_{#1}}
\newcommand{\spec}[1]{\mathrm{Spec}(#1)}
\newcommand{\mspec}[1]{\mathrm{MSpec}(#1)}

\graphicspath{{./figures/}} %note trailing /

%\usepackage{showkeys}

\begin{document}

\section{Basics, 8/22/16} We'll let $k$ stand for a field, e.g. ($\rats,
\reals, \cmplx, \ffield q, \cmplx(x_1, ..., x_n)$). Let $\polyring k 1 n$ be
the polynomial ring in $n$ generators (ignoring grading). Let $\polyring k 0 n
= \graded k n$ be the graded ring in $n + 1$ generators, where
$\polyring{k}{0}{n}_d$ is the vector space of homogeneous polynomials of
degree $d$.

As a convention, we'll write $S := \polyring k 0 n$, and $S_d :=
\polyring{k}{0}{n}_d$. Note that we have a map $\mu: S_d \times S_e
\rightarrow S_{d + e}$ given by multiplication.

\subsection{Affine Algebraic Geometry} In affine algebraic geometry, we study
finitely generated modules over $S$ (think of this as a local picture).

\subsection{Projective algebraic geometry} Studies finitely generated graded
modules over $S$ (think of this as a global picture). This is analogous to
vector bundles on compact manifolds.

\subsection{Ideals in Polynomial Rings}
\begin{thm}[Hilbert's Basis Theorem]
If $R$ is Noetherian, $R[x]$ is.
\end{thm}

\begin{definition} $R$ is noetherian if every submodule $N \subset M$ of a
finitely generated module $M$ is also finitely generated. This is equivalent
to all ideals being finitely generated. \end{definition}

Given $f \in \polyring k 1 n$, consider $V(f) := \{ \uline{a} = (a_1, ...,
a_n) \in k^n | f(\uline{a}) = 0\}$, the hypersurface in $k^n$ associated to
$f$. Given an ideal $I \subset \polyring k 1 n$, then $V(I) = V(f_1) \cap
\cdots \cap V(f_n)$ where $I = \langle f_1, ..., f_n \rangle$, by the Hilbert
basis theorem.

\begin{thm}[Nullstellensatz] The map $m : k^n \rightarrow \mspec {\polyring k
1 n}$ given by $(a_1, ..., a_n) \rightarrow \langle x_1 - a_1, ..., x_n - a_n
\rangle = \ker(\polyring k 1 n \rightarrow k, f \rightarrow f(\uline{a})) = \{
f | f(\uline{a}) = 0 \}$ is bijective, if $k = \bar{k}$.  \end{thm}

\begin{rmk} When $n = 1$, $k[x]$ is a PID so any maximal $m = \langle f
\rangle$. But $k$ algebraically closed iff any $f$ factors iff every max ideal
is of the form $m = \langle x - a \rangle$.  \end{rmk}

\begin{cor}   Let $I$ be any proper ideal in $S$, then $V(I) \neq \emptyset$.
In fact, there is a bijection between $V(I)$ and the set of maximal ideals
containing $I$. \end{cor}

\begin{cor}   Let $I = \langle f_1, ..., f_n \rangle$ and $V(I) = \emptyset$.
Then $1 \in I$, so there exists $g_i$ such that $1 = \sum f_i g_i$. \end{cor}

\begin{definition} $X \subset k^n$ is \uline{algebraic} if $X = V(I)$ for some
$I$. \end{definition}

Given an algebraic subset, define $I(X) := \{ f | f(\uline{a}) = 0, \forall
\uline{a} \in X \}$. This function satisfies $I(V(I)) = \sqrt{I}$.

\begin{proof}   Given $I = \langle f_1, ..., f_n \rangle$, $X = V(I)$. Take $g
\in I(V(I)).$ Consider $I = \langle f_1, ..., f_n, gx_{n + 1} - 1 \rangle$, an
ideal of $\polyring k 1 {n + 1}$. It follows that $V(I) = \emptyset$, so $1 =
\sum g_i f_i + h(g x_{n + 1} - 1)$. Now take $g = x_{n + 1}$, so that \[ 1 =
\sum g_i(x_1, ..., x_n, 1/g) f_i(x_1, ..., x_n).\] Clearing denominators gives
us $g^N$, for some $N$, as a linear combination of the $f_i$, so $g^N \in I$,
i.e. $g \in \sqrt{I}$. \end{proof}

\section{Overview, 8/24/16}

\begin{definition}   $J \subset \polyring k 1 n$ is geometric ideal if $J =
I(S)$ for some $S \subset k^n$. \end{definition}

The nullstellensatz tells us that $J$ is geometric if and only if $J$ is a
radical ideal, giving us a bijection between \begin{enumerate} \item algebraic
sets of $k^n$ \item geometric ideals of $\polyring k 1 n$ \item reduced
quotient $k$-algebra $R$ with $\polyring k 1 n \twoheadrightarrow R$ (meaning
no nilpotents) \end{enumerate}

The first bijection is inclusion reversing.  We also have a bijection between
\begin{enumerate}
\item affine varieties
\item prime ideals
\item $k$-algebra quotient domains $A$ with $\polyring k 1 n \twoheadrightarrow A =: k[X]$.
\end{enumerate}

From the original nullstellensatz, we have a bijection between
\begin{enumerate}
\item points $x \in k^n$
\item max ideals
\item field quotients $\polyring k 1 n \twoheadrightarrow k$
\end{enumerate}

\begin{definition}
If $X \subset k^n$ is a variety, then
  \begin{itemize}
    \item $k(X) :=$ field of fractions of $k[X]$
    \item $\dim X := \mathrm{tr. deg.} (k(X)/k)$
  \end{itemize}
\end{definition}

There can't really be a good definition of dimension due to components.

\subsection{Cubics in the plane}

\textbf{Example}. Consider cubics in the plane in $W$-normal form: $y^2 - (x -
r_1)(x - r_2)(x - r_3)$. For example, $f(x,y) = y^2 - x^3$. Then $X = V(f)$
is irreducible by Eisenstein's criterion (?). We have an embedding $k[X] =
k[x,y] \langle y^2 - x^3 \rangle$ into $k[t]$ by sending $x, y \rightarrow t^2,
t^3$, i.e. $k[X] \iso k[t^2, t^3]$. However, $k(X) = k(t)$ is the field of
rational functions in one variable. Geometric intuition: $f$ has well-defined
tangent lines except at the origin.

Let $g(x,y) = y^2 - x^2(x + 1)$, and $Y = V(g)$. We have an embedding $k[Y]
\xhookrightarrow{} k[t]$ by $x, y \rightarrow t^2 - 1, t(t^2 - 1)$. Hence
$k[Y] \iso k[t^2 - 1, t(t^2 - 1)]$. Again, $k(Y) \iso k(t)$. A commutative
algebraic intuition for this is that $k[X]$ and $k[Y]$ are not integrally
closed, and $k[t]$ is actually their integral closure.

If we let $h_\lambda := y^2 - x(x + 1)(x - \lambda)$, for $\lambda \neq 0,
-1$. Let $Z_\lambda := V(h_\lambda)$, then it turns out that $k(Z_\lambda)
\not \iso k(t)$. We will actually prove this by constructing differentials on
this curve. Moreover, up to a finite group acting on the $\lambda$,
$k(Z_{\lambda_1}) \not \iso k(Z_{\lambda_2})$ if $\lambda_1 \neq \lambda_2$.

\begin{rmk} It turns out $h_\lambda$ has no common zeros with its partial
derivatives. If they did, calculations can show that $\lambda = 0, -1$, a
contradiction. Hence by the implicit function theore, $h_\lambda$ is a
manifold. It turns out that $Z_\lambda$ being a manifold is related to the
fact that $k(Z_\lambda) \neq k(t)$. \end{rmk}

At first glance, this isn't entirely true though: $W = V(y - x^3)$ is also a
manifold and $k(W) \iso k(X)$. However, it turns out that $W$ has a singular
point at infinity.

\begin{definition}   Let $X = V(P)$, some prime ideal $P$, $P = \langle f_1,
..., f_m \rangle$. If $\uline{a} = (a_1, ..., a_n)$, $J = \left(
\dfrac{\partial f_i}{\partial x_j} \right)$. We say $X$ is \uline{non-
singular} at $\uline{a}$ if $J(\uline{a})$ has rank $n - \dim X$.
\end{definition}

\begin{thm} At all $x \in X$, $\mathrm{rank} J \leq n - \dim X$. Equality
holds away from an algebraic subset $Y \subset X$, which has measure 0, a
consequence of the Zariski topology. \end{thm}

\subsection{Birational geometry} In birational geometry, we study the
classification of fields $K$ of finite transcendence degree over $k =
\bar{k}$. The idea is that if $K$ has transcendence degree $n$, meaning there
are $x_1, ..., x_n$ such that $K/k(x_1, ..., x_n)$ is a finite field
extension. Consider the ring of integers of both fields, and let the ring of
integers of the larger field be $A$.

\begin{thm} $A$ is a finitely generated $k$-algebra. \end{thm}

It turns out that $A$ corresponds to a \uline{normal variety} $X \subset k^N$
(a normal affine model of $K$). The question then arises: can you find $x_1,
..., x_n \in K$ such that the model $X$ (and points at infinity) is non-
singular. Hironaka answered this question affirmatively for characteristic
zero, and it is unknown in characteristic $p$.

\end{document}
